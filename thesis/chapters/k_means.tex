\chapter{k-means approach for cluster analysis}

%Esistono attività di coprescrizione, analisi traiettorie
%Ie: normix come strumento di profilazione (es. età), layer, modelli di utilizzo del prodotto: spazio, tempo, fattore di struttura (antibiotici più prescritti)
%Cluster in base ai medici, se il medico è un fedele prescrittore: esistono tipologie (analisi fattoriale?) k-means, variabili attive (antibiotici) e illustrative
%Se esistono comportamenti che classificano i medici: medici, prescrizioni, tempo 
%Medici | antibiotici | numero prescrizioni | età pazienti | città
%3 di queste per capire se i modelli di comportamento
%Cluster annidate? Numero cluster? Iniziare con 5 gruppi, 100 iterazioni

%# trajectories analysis for co-prescription activities
%# antibiotics as profiling tools, with additional layers (ways to use most prescribed products)
%# factorial analysis to understand whether a GP makes constant prescriptions


A practical application of clustering consists in grouping general practitioners according to their prescriptive habits of antibiotics.

$k$-means is the best clustering algorithm, since the approach is non-hierarchical and simple robe cose

\section{Range of time}
Since the aim of clustering is identifying prescription patterns and their changes, extracting different time slices allows better insights on evolutions of trends.

To avoid the impact of seasonality, having values increasing in the winter months, data is collected and aggregated in a range of one year: the final table contains cumulative results and features, without distinction between different times of the year.

Comparisons are made selecting two photographies of data according to different years, and visualising outcomes to understand whether values have shifted cluster (general practitioners have changed habits).

Data in two years needs to be distant enough to ensure the presence of potential changes, yet consistent enough to prevent information loss. 

Selected years are 2010 and 2017, 2017 being the latest complete time range and 2010 being the first year considered for antibiotics analysis.

\section{Dataset construction}
The most important constraint while constructing the matrix is consistency of data: all general practitioners must be active in both years, to make comparison possible.

The number of constantly active doctors is 372, having respectively 228.022 and 214.316 patients in 2010 and 2017. This represent a data loss of about 75\% in both values.

Features to consider are:
\begin{itemize}
	\item Antibiotic prescriptions;
	\item Patients data;
	\item Other prescriptions data.
\end{itemize}

Antibiotic prescriptions are divided into the 8 most popular products, while patients data include sex and most common age range. Information about the total amount of prescriptions might be helpful to understand the influence of antibiotics.

The final table appears this way:
% table

\section{Features selection}

\section{Optimal number of clusters}

\section{Results}
% quanti doctors hanno cambiato cluster?