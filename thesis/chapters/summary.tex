\chapter{Summary}

This research work aims to support existing national studies on the medicine consumption in Italy and the growing problem of antibiotic resistance, through exploratory analysis and clustering on healthcare data.

After explaining the Italian sanitary system along with functioning of the pharmaceutical products' market, an overview of healthcare data is given, underlying its collection, potentialities and issues in the first chapters.

The available dataset comprehends medical data on patients with related prescriptions and diagnoses in the past 18 years in the Italian region Campania.

Since health data consists in large quantities of information, an in-depth understanding of its structure and value is required, assessing the risk of progressive information loss and constructing subsets of clean values to perform statistics in chapters 4-5.

Global analysis in chapter 6 allows to identify anomalies to focus on, focussing on antibiotic consumption in chapter 8: prescription trends are highlighted in relation with national research, explaining potential causes of unusual patterns.

The patient-doctor relationship is reconstructed in chapter 7, resulting in a consistent dataset of medical history to apply clustering algorithms and analyse co-prescriptions.

Clustering is performed according to prescribing habits based on doctors and diagnoses, detecting communities of prescribers with similar conduct, medicines preference and overprescribing.

Chapters 9-10 focus on various approaches of clustering based on antibiotic prescriptions and comparing results, using the subset of information and the obtained trends.

The scope of the research is limited to prescription patterns, yet healthcare data has many other applications: future additional work is required to give a complete insight, considering a wider spectrum of medicines and relating them to diagnoses.


