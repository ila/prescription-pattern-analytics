\chapter{Introduction}
\section{Antibiotic resistance and misuse}
Antimicrobial resistance is a rising global problem which threatens the effective prevention and treatment of an ever-increasing range of infections caused by bacteria, parasites, viruses and fungi\cite{who}. 

Microorganisms exposed to antimicrobial drugs develop the ability to defeat substances designed to kill them, making infections persist in the body due to the unsuccessful action of agents.

This issue threatens public health causing higher healthcare costs to treat patients, and potentially compromising surgeries and chemotherapy results due to the ineffectiveness of antibiotics. No one can completely avoid the risk of resistant infections, but some people are at greater risk than others (for example, people with chronic illnesses)\cite{cdc}.

Antimicrobial resistance occurs naturally over time, usually through genetic changes. However, \textbf{the misuse and overuse of antimicrobials is accelerating this process}. In many places, antibiotics are overused and misused in people and animals, and often given without professional oversight\cite{who}.

Infections such as the common cold or sore throats are often countered with antibiotics, which have no effect against viruses and could as well put patients at the risk of suffering adverse reaction\cite{bmj}.

Therefore, to minimise the development of resistance, contributing factors must be reduced, optimising the use of drugs. This work requires effective surveillance and follow-up of consumption, at a local and national level. 

To be effective in the long run, work to optimise use of antibiotics must influence the prescribing practices of individual physicians. The goal is “rational use”, i.e. the correct patient receives the correct antibiotic at the correct dose and for the correct duration of treatment, in accordance with evidence-based guidelines. Over-prescribing should be avoided without resulting in under-prescribing\cite{sweden}.

Such work should be carried out close to the prescriber, something which also requires high resolution prescription data, down to the clinic and health centre level or better still at the level of \textbf{individual prescribers}.

\subsection{In Italy}
%Inquadramento generale del problema: sistema sanitario nazionale e suo funzionamento, ruolo del medico di base e prescrizioni di farmaci etici (scenario, evoluzione del settore) 30 pagine
%Rapporti di farmindustria, siti ministero Salute, agenas, federfarma


\section{Healthcare data}
%dati, collezione, come sono raccolti, open data in italia, società (aicuvia oims health), promofarma (farmacisti, organizza informaticamente i dati e li vende → aziende farmaceutiche), agenzia delle entrate (codice fiscale), spiegare come sono organizzati ecc
%3) proprietari sistemi sw di medicina generale, millennium
%3 canali di produzione del dato (medici, farmacisti, ae) + canale derivato aicuvia

\section{Data classification}
% idc9, atc, etc

\section{Healthcare analytics}
\textbf{Healthcare analytics} is a field of growing importance which allows a deep understanding of results collected in the healthcare area. 

Extracting insights can be a complex challenge: health big data gives a huge \textit{volume} and \textit{variety} of information, therefore accessing the resources in a quick way is necessary. 

Other issues to deal with are \textit{veracity}, \textit{validity} and \textit{viability}, fundamental characteristics to ensure reliable and relevant analytics. Checking for integrity and quality can be difficult to verify without domain knowledge\cite{4vs}.

One of the various applications of this field is the change of patterns in the historical data: this research specifically focuses on \textbf{prescription pattern changes} on chronic patients, highlighting the development of some common medicines through time.

The considered healthcare data is a dump of a database created and handled by Millennium Srl\cite{millewin}, a leader company in IT services for medicine. This has been obtained through an extraction procedure on the original DB, assigning unique names to each table and column.

There are five main necessary fields for analysis\cite{DC}:
\begin{enumerate}
	\item \textbf{Spatial data}, in different granularity levels;
	\item \textbf{Personal data} of patients;
	\item \textbf{Temporal data}, in a range from 2000 to 2018, for time-series analysis;
	\item \textbf{Pharmacotherapeutic data}, classifiable according to different identification codes;
	\item \textbf{Diagnostic data}, for cross-validation of diagnoses.
\end{enumerate}

The two main risks encountered while doing analysis are information loss and inappropriate prescribing, that compromise the quality of statistics. Data may be incomplete, biased or filled with noise: another goal of analytics is to contrast incompleteness and incorrectness, obtaining coherent and clear results.
