\chapter{Introduction}
\section{Antibiotic resistance and misuse}
Antimicrobial resistance is a rising global problem which threatens the effective prevention and treatment of an ever-increasing range of infections caused by bacteria, parasites, viruses and fungi\cite{who}. 

Microorganisms exposed to antimicrobial drugs develop the ability to defeat substances designed to kill them, making infections persist in the body due to the unsuccessful action of agents.

This issue threatens public health causing higher healthcare costs to treat patients, and potentially compromising surgeries and chemotherapy results due to the ineffectiveness of antibiotics. No one can completely avoid the risk of resistant infections, but some people are at greater risk than others (for example, people with chronic illnesses)\cite{cdc}.

Antimicrobial resistance occurs naturally over time, usually through genetic changes. However, \textbf{the misuse and overuse of antimicrobials is accelerating this process}. In many places, antibiotics are overused and misused in people and animals, and often given without professional oversight\cite{who}.

Infections such as the common cold or sore throats are often countered with antibiotics, which have no effect against viruses and could as well put patients at the risk of suffering adverse reaction\cite{bmj}.

Those critical issues are worsened by the fact that at the moment there are no antibiotic drugs in development, and no trials in the past 30 years led to discoveries of new antimicrobial medicines\cite{agenziafarmaco}. 

Therefore, to minimise the development of resistance, contributing factors must be reduced, optimising the use of drugs. This work requires effective surveillance and follow-up of consumption, at a local and national level. 

To be effective in the long run, work to optimise use of antibiotics must influence the prescribing practices of individual physicians. The goal is “rational use”, i.e. the correct patient receives the correct antibiotic at the correct dose and for the correct duration of treatment, in accordance with evidence-based guidelines. Over-prescribing should be avoided without resulting in under-prescribing\cite{sweden}.

Such work should be carried out close to the prescriber, something which also requires high resolution prescription data, down at the level of \textbf{individual prescribers}.

\section{Italian National Sanitary Service}
The Italian National Sanitary Service (SSN) consists the complex of functions, activities and healthcare services offered by the State. It is based on subsidiarity, a general principle of the European Union law, stating that a central authority should have a subsidiary function, performing only those tasks which cannot be performed at a more local level\cite{oxford}.

SSN is articulated in different responsibility levels divided among the State, Regions, institutions and organisations, along with private structures and the Health Ministry, which coordinates the national sanitary plan.

Citizens benefit of healthcare services paying a related ticket\cite{ticket}, which represents the established way to contribute to expenses. It is used for:
\begin{itemize}
	\item Specialist examinations;
	\item First-aid help in non-emergency situations;
	\item Thermal care.
\end{itemize}

Sanitary assistance on the territory is free, in fact general practitioners' visits are exempted from payment of tickets (while additional services such as certificates may require a fee). A general practitioner (GP) is a way for the citizen to access SSN in terms of global care and health education.

GPs manage types of illness that present in an undifferentiated way at an early stage of development, which may require urgent intervention. Their duties are not confined to specific organs of the body, and they have particular skills in treating people with multiple health issues\cite{wonca2}. 

According to WONCA (World Organization of Family Doctors), they are responsible of supplying integrated and continuative care. Some fundamental skills and activities\cite{wonca2} to pursue this goal are:
\begin{itemize}
	\item Communication with patients;
	\item Management of the practice;
	\item Clinical tasks;
	\item Problem solving;
	\item Holistic modelling.
\end{itemize}

In Italy, GPs have a crucial role in preventing diseases, understanding the symptoms, introducing patients to therapeutical approaches and monitoring the development or regression of illnesses\cite{gp}. Primary aid is guaranteed through diagnoses, prescription, therapy and basic levels of assistance.

Visits take place at the medical office, according to methodologies established by the doctor, generically on appointment or at the patient's domicile. After a visit, a common task of the general practitioner in agreement with SSN is prescribing drugs or further medical checks through a prescription.

Prescriptions are healthcare documents that govern the plan of care for an individual patient\cite{ascpt}, consisting in written authorization to purchase a specific medicine from a pharmacy.

Drugs are dispensed according to the national guidelines\cite{ricette}, with the following regimes:
\begin{itemize}
	\item OTC (Over The Counter), not subject to medical prescription;
	\item RR (Repeatable Prescription), to be sold after presenting a prescription;
	\item RNR (Non-Repeatable Prescription), to be sold after presenting a prescription which has to be renewed each time;
	\item RL (Limitative Prescription), used in hospitals and clinics;
	\item RMR (Ministerial Tracing Prescription), for narcotics and psychotropic substances.
\end{itemize}

\subsection{Pharmaceutical products and prescriptions}  
The Italian Pharmacy Agency (AIFA) is the public institution for regulatory activity of drugs in Italy. Its main duty consists in all the activities related to the regulatory process of drugs, registering and authorising them to commercialization with a negotiated price. 

Before a drug can be sold in pharmacies among the Italian territory, it must have received the authorisation by AIFA: each medicine is subject to checks regarding chemical, pharmaceutical, biological, toxicological, and clinical aspects and researches, to see if it satisfies security and efficacy standards\cite{aic}.

After passing all the quality controls, a product is assigned an unique AIC code for it to be identified with its specific details.

AIFA guarantees uniformity and equity of the pharmaceutical system, coordinating national and local authorities such as Regions. Procedures are based on safeness, innovation and accessible healthcare: pharmaceutical costs are regulated in a context of financial compatibility with industry competitiveness, while pursuing goals of economic balance and population' safeguard.

From the year 2000 every form of participation to sanitary expenses from citizens has been abolished\cite{ticket}, yet most Regions have introduced special drugs classes with a fixed quota for each medical prescription or package, to remedy the profit deficit. 

A medicine only sold after presenting a medical prescription is defined ethical. Currently existing categories of ethical drugs according to reimbursement are\cite{classi}:
\begin{itemize}
	\item A, entirely at the expense of SSN, comprehending essential medicines and the ones for chronic diseases;
	\item H, at the expense of SSN only in a hospital environment;
	\item C, fully paid by citizens according to the brand price.
\end{itemize}
Complete lists for classes A and H are publicly available, while each single drug or active principle can be individually looked up on the official AIFA database. Prescriptions can fall into any category, while OTCs follow a C regime. 

Generic drugs have reduced costs, and to have a brand product the citizen must explicitly ask for it and pay the additional price\cite{ticket}.

\subsection{Antibiotic consumption in Italy}
Italy is the European nation with the highest antibiotic resistance mortality (~10.000 deaths every year), in terms of infection caused by resistant bacteria\cite{repubblica}. The public health relevance of this issue is high, because of the considerable epidemiologic on the population (increment of morbidity and mortality) and the heavy social burden (workdays loss, usage of diagnostic procedures)\cite{aifa2017}.

AIFA published the national report ``Antibiotic use in Italy 2017", providing consumption and expenses data at national and regional level. The report allows to identify areas of potential inappropriateness and promote a comparison between Regions with the aim of improving prescriptions and antibiotic use.

Resistance to methicillin goes up to 38\%, which is a non-negligible emergency; furthermore, people affected by MRSA (methicillin-resistant Staphylococcus aureus) have 64\% more probability of death compared to individuals who didn't develop an antibiotic resistant infection\cite{agenziafarmaco}.

Antibiotic resistant infections are diffused in all age ranges, but particularly affect the extremes (individuals aged 75 and more years). 

The problem is aggravated by the fact antibiotics are drugs sold over the counter in Italy, therefore individuals are able to get medicines according to their own judgement without a prescription by an expert --- even if their diagnosis doesn't involve an antibiotic cure.

AIFA recommends actions to raise awareness among the population\cite{aifaar}:
\begin{itemize}
	\item Using antibiotics only if prescribed by a doctor;
	\item Completing the therapy without interrupting it;
	\item Avoiding taking several antibiotics in short spans of time.
\end{itemize}

Patients are not the only ones contributing to antibiotic resistance: another aspect of it is the lack of ethical prescriptions from general practitioners and the poor control of sanitary system workers by the authorities. 

For instance, 75\% of cases is due to infections related to sanitary assistance, and the need to contrast those actions, especially in patient care structures, is rising. 

More than 85\% of doses have been provided by SSN (National Sanitary Service), both from private or public pharmacies and public sanitary structures. 90\% of those doses are due to prescriptions from general practitioners or paediatricians.

Geographical analysis confirms a greater consumption in the southern and centre regions compared to the north. Campania is the region with most antibiotic usage and the highest average costs, although values have suffered a slight decline since 2016.

There is a marked increase in consumes between seasons, in particular between the summer months and the winter ones, related to the flu symptoms peaks observed in winter during the years. A relevant part of seasonal prescriptions could be avoided, since a considerable amount of infections spreading in the cold months is viral.

OMS in the $20^{\text{word or phrase}th}$ WHO List of Essential Medicines (2017) groups antibiotics in three categories, with the aim of guiding their prescribing:
\begin{enumerate}
	\item Access, which should always be used as a first-choice treatment (penicillins with broad spectrum);
	\item Watch, antibiotics with higher risk to induce resistance (cephalosporins, macrolides);
	\item Reserve, antibiotics to be given in serious diseases after other unsuccessful alternatives, with exclusive hospital usage.
\end{enumerate}

The antibiotic classes with the most use prevalence in Italy are penicillins, comprehending amoxicillin and beta-lactose inhibitors (clavulanic acid) , macrolides and cephalosporins, having a major detach from all the other antibiotics.

The association between amoxicillin and clavulanic acid suggests a probable over-use in cases where only prescribing amoxicillin could have a minor impact on resistance with a selective spectrum of actions.

Furthermore, 40\% of prescriptions in 2017 did not involve a first-choice (access) antibiotic, with a growing tread from North to South.

Equivalent drugs usage is still a minor percentage: 70,1\% of consumption is composed by brand drugs with expired license, and Campania is again one of the regions with the lowest incidence of generic medicines.

Inappropriate consuming and antibiotic abuse can be countered only with a global \textit{one health} approach, promoting interventions for responsible use of medicines in all fields\cite{aifa2017}.

The focal point of monitoring and implementing initiatives to improve prescriptive appropriateness is represented by general practitioners, due to those being the main source of antibiotic prescriptions.

\section{Healthcare data}
Healthcare data is defined as that information used to provide, manage and/or report the services used across the entire healthcare system. Its origin is the encounter between a patient and a provider, who will record the service rendered, the conditions of the service, patient information and clinical information\cite{dataquality}.

To trim costs and maximise productivity and value, healthcare entities are turning to their data and decision support tools to validate quality initiatives: data may not be captured completely or accurately, with incorrect information or keys and missing referrals.

Recent advances in health information technology have expanded the scope of health data. Advances in health information technology have fostered the eHealth paradigm, which has expanded the collection, use, and philosophy of health data.

eHealth is a recent healthcare practice defined as a set of technological themes in health today, more specifically based on commerce, activities, stakeholders, outcomes, locations, or perspectives\cite{ehealth}.

Health information is understood and appraised among Electronic Health Records, prescribing, health knowledge managements and information systems. The main concern is the condifentiality of the data, standardised through coding techniques.

Electronic Health Records is a widespread application of big data in medicine. Patients have their own digital record which includes demographics, medical history, allergies, laboratory test results etc. Records are shared via secure information systems and are available for providers from both public and private sector\cite{datapine}.

\subsection{In Italy}
Italy has different typologies of data collection, distinguished by both the source and the practical applications. Production is gathered in three different channels, with eventual derivations.

\subsubsection{Public data}
Public data is produced by public institutions, using the national registries forms submitted by taxpayers and personal data of services users.

Ministero della Salute (Health Ministry) is the national entity to promote health as a fundamental right, and owns an open data system containing information about chronic and rare diseases, public structures and medical devices classification. 

The purpose of public datasets is mainly informative, since data is already presented in an aggregated form.

Personal data of patients is collected by Agenzia delle Entrate, having the duty of collecting taxes and revenues. Sanitary expenses are detracted from taxes, including partial costs of pharmaceutical products. 

Agenzia delle Entrate produces a preventive and consumptive balance sheet every year, elaborating results of various duties and concessions.

Another national source of healthcare information is ISTAT (National Institute of Statistics), collecting data, microdata and metadata to make analysis and classifications. Its main areas concern suicides, road accidents and mental health.

Public data consists in a small part compared to all the existing datasets concerning Italy, which are currently private.

\subsubsection{Pharmaceutical data}
Pharmaceutical data has as its source the activity of pharmacies, interacting with the market while buying and selling both OTC products and prescribed ones.

Promofarma is the main commercial society dedicated to prescriptions data collecting, offering:
\begin{itemize}
	\item Studies on pharmaceutical expense and profitability;
	\item Outsourcing services for local network handling.
\end{itemize}

It organises information to then aggregate and send it to public services, publishing private reports on consumption trends. Analysis is made according to various granularity levels.

Data is electronically transmitted to Ministero dell'Economia e delle Finanze (Economy and Finance Ministry) and Ministero della Salute (Health Ministry), and all pharmacies must monthly submit records\cite{promofarma}. About 400 millions of prescriptions are sent every year.

\subsubsection{Private service providers}
Private healthcare data is produced by individuals (i.e.\ general practitioners) or structures agreeing to use a proprietary storage system, submitting information to companies.

Instances of private healthcare industry services providers are:
\begin{itemize}
	\item IMS Health, one of the largest vendors of data, collecting electronic health records and prescription data:
	\begin{itemize}
		\item IQVIA integrates data science with human science, automatizing health processes with information services;
	\end{itemize}
	\item Complion, a document management and workflow platform for clinical research sites;
	\item Millenium, offering a suite of proprietary systems of family medicine applications on a national level.
\end{itemize}

Those channels make a preprocessing action, outputting derived data to be linked with evidence and analytics for concrete decisions making.

Results are sold to research centres and pharmaceutical companies to make custom analytics based on specific needs and desired outcomes. 

\section{Data classification}
Analytical processes require precise standards concerning data integrity and availability, since decision-making processes must be reliable and transparent.

Healthcare data, during the collection and parsing processes, needs to be classified according to national or international standards, in a way that information cannot be misinterpreted and fields can be mapped to wider categories. Some standardization exists in the way data is captured.

\subsection{ICD}
ICD (International Classification of Diseases) is the foundation for global identification of health trends and statistics, and the international standard for reporting diseases and health conditions. It is the diagnostic classification standard for all clinical and research purposes\cite{who}, maintained by the World Health Organization (WHO).

ICD defines the universe of diseases, disorders, injuries and other related health conditions, listed in a comprehensive, hierarchical fashion that allows for: 
\begin{itemize}
	\item Easy storage, retrieval and analysis of health information for evidenced-based decision-making;
	\item Sharing and comparing health information between hospitals, regions, settings and countries;
	\item Data comparisons in the same location across different time periods\cite{whoicd}.
\end{itemize}

The ICD is revised periodically and is currently in its 10th version, while in Italy the latter has only been adopted to classify death causes\cite{icdit}. Until a complete upgrade, the diagnostic system is standardised using ICD-9.

\subsubsection{ICD-9}
ICD-9, officialized in 1978, is the 9th version of the International Classification of Diseases, ordering diseases and traumas in groups according to defined criteria, allowing a common language to code information related to morbidity and mortality for comparisons and statistics\cite{icdit}.

ICD9-CM is an adaption to ICD-9 used in Italy to assign diagnostic and procedure identifiers, providing additional morbidity detail. Diagnoses are extended with codes for surgical, diagnostic and therapeutical procedures. 

It is composed by a group of three digits followed by up to two optional ones adding further details, separated with a dot:
\begin{enumerate}
	\item The first number (001-999) represents the macro-category based on the type of the disease or the injury they describe;
	\item The second group provides more specific information about the type, location, and severity of the disease or injury.
\end{enumerate}

There are also two sets of alphanumeric codes in ICD-9-CM. E-codes describe external causes of injury, while V-codes describe factors that influence health status and/or describe interactions with health services\cite{icd9en}.

Example: 414.12, falling in diseases of the circulatory system (390-459), specifically into ischaemic heart disease (410-414).
\begin{itemize}
	\item 414: other forms of chronic ischaemic heart disease;
		\item 1: aneurysm and dissection of heart;
			\item 2: dissection of coronary artery.
\end{itemize}

\subsection{ATC}
The Anatomical Therapeutic Chemical Classification System (ATC) is a drug coding system adopted worldwide, controlled by World Health Organisation.

Medicines are divided into different groups according to to the organ or system on which they act, their therapeutic intent or nature, and the drug's chemical characteristics. Different brands share the same code if they have the same active principle and indications.

One single drug can have multiple codes, since ATC also comprehends instructions regarding administration or use, and a code can represent more than one active ingredient.

The ATC classification is composed by seven alphanumeric symbols split into five adjacent hierarchical sets, defined levels and having the following structure\cite{atclevels}:
\begin{enumerate}
	\item One letter indicating the anatomical/pharmacological main group among 14;
	\item Two digits indicating the therapeutic subgroup;
	\item One letter indicating the pharmacological subgroup;
	\item One letter indicating the chemical subgroup;
	\item Two digits indicating the chemical substance.
\end{enumerate}

The ATC system also includes many defined daily doses (DDDs). This is a measurement of the assumed average maintenance dose per day for a drug used for its main indication in adults.

Alterations in ATC classification can be made when the main use of a product has clearly changed, and when new groups are required to accommodate new substances or to achieve better specificity in the groupings. Changes are twice annually submitted to the WHO official database.

Example: C03BA12.
\begin{itemize}
	\item C: cardiovascular system;
		\item 03: diuretics;
			\item B: low-ceiling diuretics, excluding thiazides;
				\item A: sulfonamides, plain;
					\item 12: Clorexolone.
\end{itemize}

\subsection{AIC}
AIC represents authorisation to admission to commerce of a medicine, and is a 9-digit code conceded by AIFA after a careful check of safeness and efficacy. It is a sort of ``identity card" of the drug, since it contains the essential characteristics defining it\cite{aicdef}. Different brands are identified by different AIC.

AIC establishes:
\begin{itemize}
	\item The drug name;
	\item Its composition (active principle);
	\item Description of the fabrication method;
	\item Therapeutical instructions, contraindications and adverse reactions;
	\item Dosage and way of administration;
	 \item Conservation measures;
	 \item Characteristics of the product and its packaging;
	 \item Brochure;
	 \item Risks evaluation for the environment.
\end{itemize}

Every possible modification of those characteristics involves a further request for authorization to AIFA. Official databases such as Fedefarma contain information related to every product, along with its eventual expire of authorisation.

\subsection{Privacy concerns}
Healthcare is a highly regulated industry where the ability to achieve, maintain and efficiently demonstrate regulatory compliance improves the organizations' overall security posture, allowing them to focus on patient care and improved outcomes. 

When implemented with complimentary solutions, data classification can play a pivotal role in managing regulated data with precision, effectiveness and a level of efficiency that allows healthcare organizations the opportunity to properly focus on their core mission\cite{privacy}.

The General Data Protection Regulation (GDPR) recognises data concerning health as a special category of data, and provides a definition for health data for data protection purposes. Specific safeguards for personal health data and for a definitive interpretation of the rules that allows an effective and comprehensive protection of such data have been addressed by the GDPR. 

Processes that foster innovation and better quality healthcare need robust data protection safeguards in order to maintain the trust and confidence of individuals in the rules designed to protect their data\cite{gdpr}.

\section{Healthcare analytics}
\textbf{Healthcare analytics} is a field of growing importance which helps understanding the statistical perspective of results collected in the healthcare area. 

Extracting insights can be a complex challenge: health big data gives a huge \textit{volume} and \textit{variety} of information, therefore accessing the resources in a quick way is necessary. 

Other issues to deal with are \textit{veracity}, \textit{validity} and \textit{viability}, fundamental characteristics to ensure reliable and relevant analytics. Checking for integrity and quality can be difficult to verify without domain knowledge\cite{4vs}.

One of the possible applications, given the concerning issue of antibiotic resistance, is explaining the situation through statistics and trends, obtaining practical results alongside theoretical scientific research. 

Big data can assess the appropriateness of prescribing through the existing classification systems, comparing their patterns within time ranges. A general view is essential to identify specific unusual changes, extending national studies with a perspective centred on products and reduced geographical areas.

There are five main necessary fields for analysis\cite{DC}:
\begin{enumerate}
	\item \textbf{Spatial data}, in different granularity levels;
	\item \textbf{Personal data} of patients;
	\item \textbf{Temporal data}, for time-series analysis;
	\item \textbf{Pharmacotherapeutic data}, classifiable according to different identification codes;
	\item \textbf{Diagnostic data}, for cross-validation of diagnoses.
\end{enumerate}

The two main risks encountered while doing analysis are information loss and inappropriate prescribing, that compromise the quality of statistics. Data may be incomplete, biased or filled with noise: another goal of analytics is to contrast incompleteness and incorrectness, obtaining coherent and clear results.
