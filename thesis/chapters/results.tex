\chapter{Assessments of results and future directions}

The research ultimately accomplishes its key objective, confirming AIFA reports on antibiotic resistance in Italy: \textit{the amount of yearly prescriptions grew of almost 50 000 in the last 10 years}, despite the number of patients remaining stable and the lack of general spread of illness. 

The same outcome is also shown by the reconstruction of patient journeys, analysing first-time prescriptions and diagnoses among the complete history of patients.

Graph analytics gives an additional insight on co-prescriptions and popularity of antibiotics, identifying ones prescribed together and comparing values to the amount of patients of each doctor.

The most popular antibiotic corresponds to \textbf{amoxicillin and beta-lactamase inhibitors}, active principle on which is based \textbf{Augmentin} --- both trends are main subjects of the significant increase.

Prescriptive patterns among general practitioners allow to distinguish between classes of habits, characterised by preference for Augmentin or Normix, and the amount of antibiotics doctors tend to give.

Additional analysis should be performed on variations through time, understanding the relationship between number of patients and number of prescriptions to check whether general practitioners are effective over-prescribers. It is necessary to consider the composition effect, since values can be affected by patients switching doctors.

Those results were obtained only after an extensive preprocessing of data, optimisation of queries and analysis on progressive information loss: healthcare data is in fact difficult to handle without considering its quantity and variety, both on relational and graph models.

This work is only the beginning of a more detailed research: restricting the domain to antibiotics implies the need of future analytics expanding the scope, for instance to other concerns raised while making global statistics such as the abnormal prescription of digestive trait medicines.
